Tato diplomová práce se zabývá % TODO

První kapitola se věnuje definici odborným pojmům z logistiky, a to především z odvětví, které se zabývá plýtváním. 

V následující kapitole se nachází teoretický popis metod, které jsem použila pro datovou analýzu. Jedná se o metody pro selekci příznaků a metodu GUHA. Dále jsou v~kapitole posány důležité pojmy týkající se korelační analýzy. Závěr kapitoly je věnován popisu použitých nástrojů.

Ve třetí kapitole je definován pojem shrink a jeho klasifikace v~literatuře a v~obdržených datech.

Další kapitola se zabývá popisem obdžených dat vybrané společnosti a přípravě vzorku pro další analýzy.

V páte kapitole je popsán report, který vizualizuje obdržená data.

Čestá kapitola obsahuje návrh řešení pro kategorizaci shrinkovaných produktů pomocí korelační analýzy. V kapitole je uveden postup analýzy a popis implementace v jazyce Python. Konec kapitoly je věnován ukázce výsledků této metody.

Poslední kapitola analyzuje data pomocí procedury GUHA a metody 4ftMiner. V této kapitole bylo vysloveno několik hypotéz a následně byla ověřována jejich platnost.

% - Data
%     - jak jsou data uložená v~DB u zákazníka
%         - provázané
%     - SQL příkazy
%     - výběr proměnných
%     - target hodnoty - cost, množství
%     - produktová Hierarchie
% - Vymazaní outlierů
%     - outlier metody
%     - businessově

% - Co vysvetluje target
%     - Miner
%     - PCA

% - Korelační analýza mezi produkty v~rámci kategorie
%     - korelace 
% - Rozčlenění produktů

% - Vizualizace dat
%     - Jak funguje PBI
%     - Seznam metri