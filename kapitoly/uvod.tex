Tato diplomová práce se zabývá analýzou dat vybrané společnosti. Cílem práce je vytvořit metodický návodný postup, jak může společnost přistupovat k~tzv. shrinkům.
Jedná se o záznamy o produktech, které z~různých důvodů nemohly být prodány a kvůli tomu společnost přišla o zisk a vynaložila zbytečné náklady související s~nákupem a logistikou dotčených produktů. 
Pro redukci shrinků je zapotřebí zjistit, jaké jsou možné příčiny jejich vzniku.
Jako způsob ověření možných důvodů pro vznik shrinku jsou k~dispozici data vybrané společnosti. Na datech budou tyto hypotézy ověřeny. 

První kapitola se věnuje definici odborných pojmů z~logistiky, a to především z~od-\\větví, které se zabývá plýtváním. Popsány jsou tři hlavní typy plýtvání, dále jsou představeny možné zdroje plýtvání v~logistických procesech. Dále je v~kapitole definován pojem shrink a jeho klasifikace v~literatuře a v~obdrže-\\ných datech vybrané společnosti.


V následující kapitole se nachází teoretický popis metod, které jsem použila pro datovou analýzu. Jedná se o metody pro selekci příznaků a metodu GUHA. Dále jsou v~kapitole popsány důležité pojmy týkající se korelační analýzy. Závěr kapitoly je věnován popisu použitých nástrojů.

Další, třetí kapitola popisuje způsob získání dat vybrané společnosti. Následuje popis jednotlivých databázových tabulek a jejich sloupců. Zbylá část kapitoly se zabývá přípravou vzorku pro další analýzy. Tedy kterou částí dat se zabývat na základě četnosti a metod pro selekci proměnných. Prozkoumány jsou vztahy mezi jednotlivými příznaky.

Ve čtvrté kapitole je popsán business intelligence report vytvořený v~ aplikaci Power BI, který vizualizuje obdržená data. První část obsahuje popis jednotlivých stránek interaktivního reportu a popis použitých vizuálů. Druhá část je věnována závěrům, které z~reportu vyplývají.

Pátá kapitola obsahuje návrh řešení pro kategorizaci shrinkovaných produktů pomocí korelační analýzy. Na základě vztahu mezi výší shrinku a hodnotou tržeb jsou produkty rozděleny do několika kategorií. Každá kategorie nabízí způsob, jak má společnost s~ produkty nakládat, aby snížila shrink. V~kapitole je uveden postup analýzy a popis implementace v~jazyce Python. Konec kapitoly je věnován ukázce výsledků této metody.

Poslední kapitola analyzuje data pomocí procedury GUHA a metody 4ftMiner. V~této kapitole bylo vysloveno několik hypotéz a následně byla ověřována jejich platnost. V~závěru kapitoly je předvedeno použití metody na konkrétních produktech, u kterých se nepodařilo vysvětlit shrink pomocí korelační analýzy.

V závěru práce jsou stručně popsány jednotlivé kroky analýzy a zjištěné výsledky. Jsou diskutovány možnosti pro navázání na tuto práci s ohledem na budoucí směřo-\\vání v~oblasti analýzy dat ve vybrané společnosti.

% - Data
%     - jak jsou data uložená v~DB u zákazníka
%         - provázané
%     - SQL příkazy
%     - výběr proměnných
%     - target hodnoty - cost, množství
%     - produktová Hierarchie
% - Vymazaní outlierů
%     - outlier metody
%     - businessově

% - Co vysvetluje target
%     - Miner
%     - PCA

% - Korelační analýza mezi produkty v~rámci kategorie
%     - korelace 
% - Rozčlenění produktů

% - Vizualizace dat
%     - Jak funguje PBI
%     - Seznam metri