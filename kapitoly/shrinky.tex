\chapter{Shrink}

Cílem této práce je analyzovat shrinky produktů, které byly zaznamenány v datech dané společnosti. V následující části je vysvětlen pojem shrink a popsány kategorie, které vybraná společnost rozeznává.

\section{Definice}

Slovem shrink se označuje ztráta zisku z neuskutečněného prodeje hotového produktu. Tento produkt je vyroben, či naskladněn, ale z nějakého důvodu nemohl být prodán zákazníkovi. Tímto důvodem může být například poničení produktu, jeho ztráta nebo prošlá doba spotřeby. Za shrink produktu lze označovat i stav, kdy cena produktu je neplánovaně snížena v důsledku zmíněných důvodů. Shrinkem je potom rozdíl plánované prodejní ceny a ceny, za kterou byl produkt skutečně prodán.\cite{bib:DefShrink}

% Cílem každé společnosti by mělo být tyto shrinky minimalizovat, protože kvůli shrinkům firma přichází o zisk. 


\section{Typy shrinků}

Vybraná společnost rozlišuje ve svých datech tři kategorie shrinku -- inventory, damages, price downs. Dále se budu věnovat popisu jednotlivých typů.

\subsection*{Inventory}

Pojem inventory, který lze do češtiny přeložit jako inventář, sdružuje všechny shrinky týkající se změn v inventáři, tj. stavu zásob.

\begin{table}[]
    \begin{tabular}{lll}
    ID (motive\_type) & Název             & Popis \\
    \hline
    924          & inventory+             & Kladné připsání zboží během inventury.      \\
    923          & inventory-             & Záporné odepsání zboží během inventury.      \\
    922          & vyrovnání s inventurou & Velká inventura skladu (zahrnuje položky )      \\
    16           & oprava inventury       &       \\
    15           & castecna inventura     &       \\
    25           & neuzn. reklamace DC    &       \\
    2            & pertes merchandise     &       \\
    147          & inventura              &      
    \end{tabular}
\end{table}


\subsection*{Damages}

Do kategorie damages (česky škody) jsou řazeny zbylé důvody k odstranění produktu z prodeje, z důvodu degradace produktu. 

\begin{table}[]
    \begin{tabular}{lll}
    ID (motive\_type) & Název                  & Popis \\
    \hline
    3            & broken                 &       \\
    4            & out of date            &       \\
    6            & potravinova banka      &       \\
    8            & Pet dary               &       \\
    32           & zakaznicka reklamace   &       \\
    64           & kompostery             &       \\
    31           & neupl. reklamace DC    &       \\
    34           & internal use (err)     &       \\
    28           & zniceno rozmrazenim    &       \\
    \end{tabular}
\end{table}


