\chapter{Shrink}

Cílem této práce je analyzovat shrinky produktů, které byly zaznamenány v datech dané společnosti, a zjistit příčiny jejich vzniku. V následující části je vysvětlen pojem shrink a popsány kategorie, které vybraná společnost rozeznává ve svých datech.

\section{Definice}

Slovem shrink se označuje ztráta zisku z neuskutečněného prodeje hotového produktu. Tento produkt je vyroben, či naskladněn, ale z nějakého důvodu nemohl být prodán zákazníkovi. Tímto důvodem může být například poničení produktu, jeho ztráta nebo prošlá doba spotřeby. Za shrink produktu lze označovat i stav, kdy cena produktu je neplánovaně snížena v důsledku zmíněných důvodů. Shrinkem je potom rozdíl plánované prodejní ceny a ceny, za kterou byl produkt skutečně prodán.\cite{bib:DefShrink}

% Cílem každé společnosti by mělo být tyto shrinky minimalizovat, protože kvůli shrinkům firma přichází o zisk. 


\section{Typy shrinků}

Vybraná společnost rozlišuje ve svých datech tři kategorie shrinku -- inventory, damages, price downs. Dále se budu věnovat popisu jednotlivých typů v rámci kategorií. Každý typ má přiřazeno jednoznačné identifikační číslo, podle kterého je zaznamenáván v databázi. Z důvodů anonymizace dat v práci nejsou uvedené přesné hodnoty těchto ID, namísto toho jsou uvedeny názvy, které  definují shrinky.

\subsection*{Damages}

Do kategorie damages (česky škody) jsou řazeny zbylé důvody k odstranění produktu z prodeje z důvodu degradace produktu. V následující tabulce \ref{tab:sh:dam} jsou vypsané všechny typy, které moho být evidovány.

\begin{table}[hbtp!]
    \caption{Přehled jednotlivých typů shrinků z kategorie damages.}
    \label{tab:sh:dam}
    \begin{tabular}{ p{4cm} p{10.5cm}}
         Název             & Popis \\
    \hline
                Poškození               & Odpis zboží, které bylo poškozené. Např. nedopečené, spálené, špatně vyrobené nebo poškozené zaměstnancem nebo zákazníkem (kdy nelze uplatnit reklamaci na zákazníka.)       \\
                Prošlé a zkažené zboží  & Odpis zboží, kterému prošla doba spotřeby (v případě výrobků, kde je datum uvedené), zkažené či shnilé zboží (ovoce, zelenina) nebo ztvrdlé pečivo.       \\
                Potravinová banka       & Odpis potravinářského zboží, které bylo darováno potravinovým bankám. Jedná se o produkty, které nebylo možné zařadit znovu do oběhu.    \\
                Zvířecí útulky          & Odpis potravinářského zboží, které bylo darováno do útulků zvířat. Jedná se o produkty, které nebylo možné zařadit znovu do oběhu.          \\
                Uznané zákaznické \par reklamace \strut  & Odpis zboží, které zákazník reklamoval a reklamace byla uznána, ale zároveň nelze toto zboží reklamovat u dodavatele.      \\
                Neupl. reklamace \par distribučního centra \strut   &  Odpis zboží, které fyzicky nedorazilo z distribučního centra a nebylo možné ho reklamovat z důvodu nesplnění limitu pro vytvoření reklamace na distribučním centru. Také obsahuje odpisy neprodaných EXIT položek po ukončení výprodeje.     \\
                Kompostéry              & Odpis zboží, které je prošlé nebo poškozené a které prodejna zlikviduje v kompostéru.       \\
                Poškozeno vnějšími \par vlivy \strut %(internal use (err)) 
                                        & Odpis zboží, které bylo poškozeno nebo zničeno vlivem třetí strany (výbuch, vytopení, poškození majetku vloupáním) nebo přírodními živly. Zboží se tedy na prodejně nenachází a nemůže proto být zlikvidováno.      \\
                Zničeno  rozmražením    &       \\
    \end{tabular}
\end{table}

\subsection*{Inventory}

Pojem inventory, který lze do češtiny přeložit jako inventář, sdružuje všechny shrinky týkající se změn v inventáři, tj. stavu zásob či inventury. V tabulce \ref*{tab:sh:inv} se nachází přehled všech evidovaných typů.

\begin{table}[hbtp!]
    \caption{Přehled jednotlivých typů shrinků z kategorie inventory.}
    \label{tab:sh:inv}
    \begin{tabular}{ p{4cm} p{10.5cm}}
     Název             & Popis \\
    \hline
              Inventura$+$             & Kladné připsání zboží během inventury.      \\
              Inventura$-$             & Záporné odepsání zboží během inventury.      \\
              Inventura              & Velká inventura skladu.     \\
              Oprava inventury       & Dodatečné opravy, které bylo třeba provést po dokončení velké inventury.      \\
              Částečná inventura     & Odpis, nebo naskladnění zboží při inventuře položek.      \\
              Neuznané reklamace \par distribučního centra  \strut &  Odpis zboží, které bylo fyzicky dodané distribučním centrem na prodejnu, ale prodejna jej vrátila, ale distribuční centrum vratku neuznalo.     \\
              Inventura              & Starší verze ID používaného pro inventuru.\\
              Partes merchandise     & Odpis prokazatelně ukradeného zboží nebo i ztraceného zboží.      \\
    \end{tabular}
\end{table}


\subsection*{Price down}

