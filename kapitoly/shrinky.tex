\chapter{Shrink}

Cílem této práce je analyzovat shrinky produktů, které byly zaznamenány v datech dané společnosti. V následující části je vysvětlen pojem shrink a popsány kategorie, které vybraná společnost rozeznává.

\section{Definice}

Slovem shrink se označuje ztráta zisku z neuskutečněného prodeje hotového produktu. Tento produkt je vyroben, či naskladněn, ale z nějakého důvodu nemohl být prodán zákazníkovi. Tímto důvodem může být například poničení produktu, jeho ztráta nebo prošlá doba spotřeby. Za shrink produktu lze označovat i stav, kdy cena produktu je neplánovaně snížena v důsledku zmíněných důvodů. Shrinkem je potom rozdíl plánované prodejní ceny a ceny, za kterou byl produkt skutečně prodán.\cite{bib:DefShrink}

% Cílem každé společnosti by mělo být tyto shrinky minimalizovat, protože kvůli shrinkům firma přichází o zisk. 


\section{Typy shrinků}

Vybraná společnost rozlišuje ve svých datech tři kategorie shrinku -- inventory, damages, price downs. Dále se budu věnovat popisu jednotlivých typů.

\subsection*{Inventory}

Pojem inventory, který lze do češtiny přeložit jako inventář, sdružuje všechny shrinky týkající se změn v inventáři, tj. stavu zásob.

\begin{table}[]
    \caption{Tabulka}

    \begin{tabular}{p{1cm} p{4cm} p{9cm}}
    ID  & Název             & Popis \\
    \hline
    924          & inventory+             & Kladné připsání zboží během inventury.      \\
    923          & inventory-             & Záporné odepsání zboží během inventury.      \\
    922          & inventura              & Velká inventura skladu.     \\
    16           & oprava inventury       & Dodatečné opravy, které bylo třeba provést po dokončení velké inventury.      \\
    15           & částečná inventura     & Odpis, nebo naskladnění zboží při inventuře položek.      \\
    25           & neuznané reklamace DC    &  Odpis zboží, které bylo fyzicky dodané distribučním centrem na prodejnu, ale prodejna jej vrátila, ale distribuční centrum vratku neuznalo.     \\
    2            & partes merchandise     & Odpis prokazatelně ukradeného zboží nebo i ztraceného zboží.      \\
    147          & inventura              & Starší verze ID používaného pro inventuru.
    \end{tabular}
\end{table}


\subsection*{Damages}

Do kategorie damages (česky škody) jsou řazeny zbylé důvody k odstranění produktu z prodeje, z důvodu degradace produktu. 

\begin{table}[]
    \caption{Tabulka}

    \begin{tabular}{p{1cm} p{4cm} p{9cm}}
        ID & Název             & Popis \\
    \hline
    3            & poškození (broken)                 & Odpis zboží, které bylo poškozené. Např. nedopečené, spálené, špatně vyrobené nebo poškozené zaměstnancem nebo zákazníkem (kdy nelze uplatnit reklamaci na zákazníka.)       \\
    4            & prošlé a zkažené zboží (out of date)            & Odpis zboží, kterému prošla doba spotřeby (v případě výrobků, kde je datum uvedené), zkažené či shnilé zboží (ovoce, zelenina) nebo ztvrdlé pečivo.       \\
    6            & potravinová banka      & Odpis potravinářského zboží, které bylo darováno potravinovým bankám. Jedná se o produkty, které nebylo možné zařadit znovu do oběhu. (Ke každému předání se archivuje předávací protokol.)     \\
    8            & zvířecí útulky              & Odpis potravinářského zboží, které bylo darováno do útulků zvířat. Jedná se o produkty, které nebylo možné zařadit znovu do oběhu.  (Ke každému předání se archivuje předávací protokol.)          \\
    32           & uznané zákaznické reklamace   & Odpis zboží, které zákazník reklamoval a reklamace byla uznána, ale zároveň nelze toto zboží reklamovat u dodavatele.      \\
    31           & neupl reklamace DC    &  Odpis zboží, které fyzicky nedorazilo z distribučního centra a nebylo možné ho reklamovat z důvodu nesplnění limitu pro vytvoření reklamace na distribučním centru. Také obsahuje odpisy neprodaných EXIT položek po ukončení výprodeje.     \\
    64           & kompostéry             & Odpis zboží, které je prošlé nebo poškozené a které prodejna likviduje v kompostéru.       \\
    34           & poškozeno vnějšími vlivy(internal use (err))
    & Odpis zboží, které bylo poškozeno nebo zničeno vlivem třetí strany (výbuch, vytopení, poškození majetku vloupáním) nebo přírodními živly. Zboží se tedy na prodejně nenachází a nemůže proto být zlikvidováno.      \\
    28           & zniceno rozmrazenim    &       \\
    \end{tabular}
\end{table}




% chtěla bych Tě požádat, zda bys mohl během tohoto týdne napsat email doc. Fučíkovi (radek.fucik@fjfi.cvut.cz), že je možné mi udělit zápočet za diplomovou práci.
% V souvislosti s tím by bylo dobré Ti ukázat progress práce. Dosavadní text práce (zlomek budoucí práce) je dostupný na https://github.com/GruAna/DP. Kódy, které zpracovávám jsou na Cloudu a tam jsou i výstupy, které zatím mám. Během tohoto týdne chci dosavadní výstupy sepsat do DP, aby byly přehledněji. O krocích, které dělám jsem se bavila a bavím hlavně s Domčou. S koncem zkouškového a začátkem nového semestru (už příští týden :)) s méně předměty se diplomce a shrinkům budu věnovat naplno a chtěla bych navrhnout společné, krátké, pravidelné schůzky, abys měl opravdu přehled a kontrou nad tím co dělám a já nad sebou bič, ať mi to jde rychleji. :)
% Nevím, zda do konce týdne shrnu všechno, co jsem udělala do textu diplomky a procházet kódy a tak by bylo pro Tebe asi zbytečně náročné, takže Ti dávám své slovo, že jsem za semestr udělala část práce, případně se můžeš doptat Domči.
