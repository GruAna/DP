\chapter{Zpracování dat}

Úvod ke kapitole.... % dopsat!!!!!

\section{Popis obdržených dat}

Všechna data poskytnutá společností jsou uložena v databázi, ke které je byl zhotoven omezený přístup pro účely získání dat pro analýzy shrinku produktů společnosti. Zároveň s možností přístupu jsem obdržela i tabulku, která stručně komentuje všechny tabulky v databázi a sloupce v jednotlivých tabulkách. Celkem se v databázi nachází 412 tabulek, z nichž je potřeba vybrat ty s relevantními daty pro úlohu shrinků.

Z důvodu ochrany dat nelze uvádět přesné názvy tabulek, nicméně pro lepší orientaci v textu, každé použité tabulce přiřadím název, který odpovídá obsaženým datům v tabulce.

\subsubsection{Tabulky transakcí}
V tabulce \texttt{transakce} se nachází údaje o všech provedených transakcích, a to jak skladové transakce, tak prodeje na prodejnách. V případě prodejen jsou údaje agregované podle prodejny, konkrétního produktu a dne transakce, tzn. v této tabulce nelze rozlišit konkrétní prodeje na jednotlivých pokladnách, ale pouze souhrn za jeden den. Tabulka obsahuje údaje za posledních dvanáct měsíců.

Tabulka transakcí obsahuje 21 sloupců, jako podstatné pro analýzu jsem vybrala následující sloupce:

\begin{itemize}
    \item ID transakce, jedinečné pro každou transakci
    \item Typ transakce
    \item ID produktu, kterého se transakce týká
    \item ID skladu (včetně prodejen)
    \item Datum transakce (tzv. business datum, pokud samotná transakce proběhne až po půlnoci uvedeného dne, tak se posílá s datem z předchozího dne, neboť do toho dne businessově náleží.)
    \item ID promoce 
    \item Motive\_type z hlediska hledání shrinku produktu se jedná o klíčový sloupec, neboť obsahuje označení jednotlivých typů shrinků.
    \item Množství produktu v dané transakci
    \item Hodnota transakce v nákladové ceně
    \item Hodnota transakce v prodejní ceně včetně DPH -- v případě prodejů se jedná o skutečnou cenu, u zbylých transakcích je uvedena odpovídající cena podle ceníku.
\end{itemize}

Tabulku, která obsahuje údaje o jednotlivých prodejích na prodejnách společnosti, jsem pro účely této práce nazvala \texttt{transakce\_prodeje}. Celkem obsahuje třináct sloupců, z nichž jsem vybrala následující:

\begin{itemize}
    \item ID transakce
    \item ID produktu, kterého se transakce týká
    \item ID prodejny
    \item Datum a čas transakce 
    \item ID promoce
    \item Množství produktu v dané transakci
    \item Hodnota transakce v prodejní ceně bez DPH
    \item Hodnota DPH dané transakce
\end{itemize}

\subsubsection{Tabulky ceníků}





\section{Statistické zpracování dat}

\subsection*{Sledované údaje}



\subsection*{Období jednoho měsíce}
\subsection*{Období jednoho roku}