Cílem práce bylo analyzovat možné příčiny vzniku shrinků produktu a ověřit hypotézy na datech vybrané společnosti. Na analýzu bylo dále navázáno vytvořením návrhu řešení, které nabízí automatizovaný přístup pro redukci shrinků. Tento postup by tak mohl být použitý i na data dalších společností.

V teoretické části práce je čtenář seznámen s~odbornými pojmy z~odvětví logistiky a druhy plýtvání v~tomto oboru. Dále jsem definovala pojem shrink a nastínila problematiku tohoto pojmu. Poté jsem popsala konkrétní typy evidovaných shrinků ve vybrané společnosti. V~další části jsem sepsala princip hlavních použitých metod pro výběr proměnných a metody GUHA. Dále jsem definovala používané odborné pojmy týkající se analýzy. Teorie se také věnuje popisu nástrojů, které jsem použila při analýze. Především se jedná o popis aplikace Power BI pro vytváření interaktivních business intelligence reportů.

Nejprve jsem se seznámila s~obdrženými daty. Z~rozsáhlého množství záznamů jsem vybrala vzorový měsíc, na kterém jsem prováděla všechny analýzy. Vzhledem k~sezónnosti produktů a proměnlivosti poptávky trhu během roku nebylo vhodné provádět analýzu na všech dostupných datech nebo na celém roku. Z~databáze a externích zdrojů jsem vytipovala další data, která by pomohla vysvětlit existenci shrinků.

Stažená surová data jsem sjednotila pro další práci do samostatného datasetu. Vznikl tak dataset s~mnoha příznaky, který bylo třeba dále očistit. Odstranila jsem outliery a vybrala pouze ten typ shrinku, jehož hodnota ztracených nákladů činila nejvíce. Obdobně jsem postupovala i co se týče kategorií produktů, kterých se shrinky týkají. Z~businessové stránky problému je jasné, že ke shrinku může čas od času dojít a je třeba se soustředit pouze na ty produkty, u kterých k~němu dochází opakovaně. 
Prozkoumala jsem jednotlivé příznaky datasetu pomocí ukazatelů měřící vztahy mezi příznaky. Provedla jsem výběr proměnných pomocí analýzy hlavních komponent. Na základě zjištěných výsledků jsem označila ty příznaky, které jsou na sobě závislé a které naopak mohou pomoci vysvětlit shrink. Z~výsledků vyplynulo, že v případě hierarchických dat nemá smysl uvažovat všechny úrovně hierarchie. Na základě ukazatelů měřících závislost byla nalezena silná závislost pouze mezi příznaky, které na sobě závisí již z~povahy své definice. Konkrétní produkt a prodejna nesou část informace o výši shrinku a podílu na tržbách. Nejvíce variability v~datech nesly příznaky (s~vynecháním závislých příznaků) -- prodejna, datum transakce, počet obyvatel, šestá úroveň kategorie, kraj.

Data jsem analyzovala také pomocí interaktivního reportu. Odhaleny tak byly kategorie, kterých se shrink nejvíce týká. Nejvíce postižené jsou čerstvé výrobky. Došlo také k~porovnání evidovaných shrinků mezi jednotlivými prodejnami a regiony. Ukázalo se, že umístění prodejny nemá podle dat významný vliv na vznik shrinku. 

% Na základě těchto zjištění byly stanoveny hypotézy o souvislostech vedoucích ke vzniku shrinků. Pomocí různých metod byly tyto hypotézy potvrzeny nebo vyvráceny pro konkrétní produkty.

Provedla jsem korelační analýzu, která zkoumá závislosti mezi datasetem se shrinky a tržbami. Jedná se jak o tržby shrinkovaného produktu, tak promoční tržby ostat-\\ních produktů v~kategorii definované úrovně produktové hierarchie. Korelační ana-\\lýza takto dokáže rozdělit shrinkované produkty do několika kategorií v~závislosti na hodnotě korelačního koeficientu. Zde je důležité upozornit, že kauzalita odůvodněná hodnotou korelačního koeficientu byla businessovým rozhodnutím. V~případě pochybení by následky pro společnost nebyly fatální. Naopak se jedná o postup, který společnost může snadno ovlivnit. Společnost totiž může plánovat zásobování prodejen a své promoakce s~ohledem na kategorizaci produktů.
Postup je implementovaný v~jazyce Python jako sada funkcí, způsob použití funkcí je uveden v~ nástroji Jupyter Notebook. 
Korelační analýzu jsem spustila na obdržených datech a diskutovala jsem výsledky pro nejčastěji shrinkované kategorie.

Na základě výsledků zjištěných z vizualizačního reportu a ze vztahů mezi příznaky a cílovými sloupci jsem sestavila hypotézy, které jsem otestovala na obdržených datech metodou 4ftMiner. Touto metodou jsem zároveň analyzovala i ty produkty, u kterých se nepodařilo vysvětlit shrink korelačním vztahem s tržbami.

S ohledem  na provedená pozorování bylo navrženo, jak se chovat k~určitým produktům a jak upravovat jejich zavážené množství na prodejny nebo prodejní cenu. Eliminovat veškeré shrinky je nemožné, neboť společnost potřebuje nabízet více zboží než se skutečné prodá, aby minimalizovala výpadky zásob. Z~toho důvodu může u přebytečných jednotek produktů dojít ke zkažení nebo expiraci zboží. Nicmé-\\ně je na rozhodnutí společnosti, zda s tímto zbožím naloží ekologicky. Zlikviduje ho např. prostřednictvím kompostérů, nebo zboží poskytne potravinovým bankám, či organizacím starající se o zvířata. V~těchto institucích si již kvalitu produktů zhodnotí podle svých potřeb. Další možností je také využít část odepsaného zboží jako surovinu pro další výrobu, a to jak ve vlastním podniku, tak přeprodejem za sníženou cenu jinému subjektu.

V návaznosti na tuto práci by mohlo být dále otestováno celé portfolio vybrané společnosti.
Navržený způsob rozdělení produktů do kategorií podle jejich vztahu k~ostatním produktům by mohl být převeden do soběstačného nástroje, který nevy-\\žaduje programátorský přístup a může tak být využitý například při navrhování promoakcí produktů. Tomu samozřejmě musí předcházet businessové testování a validace.

% Bylo vyhodnoceno, kterou částí rozsáhlých dat se zabývat na základě četnosti a metod pro selekci proměnných. Prozkoumány jsou vztahy mezi jednotlivými příznaky.

