Cílem práce bylo nalézt možné příčiny vzniku shirnků produktu pro vybranou společnost a dále vytvořit prototyp řešení pro aplikování dané analýzy na data dalších společností.

V teoretické části práce jsem se seznánila s odbornými pojmy z odvětví logistiky a druhy plýtvání v tomto oboru. Dále jsem sepsala princip hlavních použitých metod pro výběr proměnných a metody GUHA. Dále jsem definovala používané odborné pojmy týkající se analýzy. Teorie se také věnuje popisu nástrojů, které jsem použila při analýze. Především se jedná o popis aplikace Power BI pro vytváření interaktivních business intelligence reportů.

Samostatná kapitola se zabývá pojmem shrink. Je uvedena jeho definice a nastíněna problematika tohoto pojmu. Pak jsem popsala konkrétní typy evidovaných shrinků dané společnosti.

Nejprve jsem se seznámila s obdrženými daty. Z rozsáhlého množství záznamů jsem vybrala vzorový měsíc s nižším počtem záznamů, na kterém jsem prováděla všechny analýzy. Vzhledem k velkému počtu záznamů a především vzhledem k sezónnosti produktů a proměnlivé poptávky trhu během roku nebylo vhodné provádět analýzu na všech dostupných datech nebo na celém roku. Z databáze a externích zdrojů jsem vytipovala další data, která by pomohla vysvětlit existenci shrinků.

Stažená surová data jsem sjednotila pro další práci do samostatného datasetu. Vzikl tak dataset s mnoha příznaky, který bylo třeba dále očistit. Odstranila jsem outliery a vybralala pouze ten typ shrinku, jehož hodnota ztracených nákladů činila nejvíce. Obdobně jsem postupovala i co se týče kategorií produktů, kterých se shrinky týkají. Z businessové stránky problému je jasné, že ke shrinku může čas od času dojít a je třeba se soustředit pouze na ty produkty, u kterých k němu dochází opakovaně. 
Prozkoumala jsem jednotlivé příznaky datasetu a označila ty, které jsou na sobě závislé a které naopak mohou pomoci vysvětlit shrink. 

Data jsem analyzovala pomocí interaktivního reportu. Odhaleny tak byly kategorie, kterých se shrink nejvíce týká. Nejvíce postižené jsou čerstvé výrobky. Došlo také k porovnání evidovaných shrinků mezi jednotlivými prodejnami a regiony. Ukázalo se, že umístění prodejny nemá podle dat významný vliv na vznik shrinku. 

% Na základě těchto zjištění byly stanoveny hypotézy o souvislostech vedoucích ke vzniku shrinků. Pomocí různých metod byly tyto hypotézy potvrzeny nebo vyvráceny pro konkrétní produkty.

Provedla jsem korelační analýzu, která zkoumá závislosti mezi datasetem se shrinky a tržbami. Jedná se jak o tržby shrinkovaného produktu, tak promoční tržby ostatních produktů v kategorii definované úrovně produktové hierarchie. Korelační analýza takto dokáže rozdělit shrinkované produkty do několika kategorií v závislosti na hodnotě korelačního koeficientu. Zde je důležité upozornit, že kauzalita vysvětlená korelací byla businessovým rozhodnutím. V případě pochybení by následky pro společnost nebyly fatální. Naopak se jedná o postup, který společnost může snadno ovlivnit.   Společnost totiž může plánovat své promoakce s ohledem na kategorizaci produktů. 
Korelační analýzu jsem spustila na obdržených datech a diskutovala jsem výsledky pro nejčastěji shrinkované kategorie  

V návaznosti na tuto práci lze otestováno celé portfolio vybrané společnosti.
Navržený způsob rozdělení produktů do kategorií podle jejich vztahu k ostatním produktům by mohl být převeden do soběstačného nástroje, který nevyžaduje programátorský přístup a může tak být využitý například při navrhování promoakcí produktů. Tomu samozřejmě musí předcházet několikanásobné testování na datech z více zdrojů, které ale nebyly k dispozizi pro tuto diplomovou práci.

% Bylo vyhodnoceno, kterou částí rozsáhlých dat se zabývat na základě četnosti a metod pro selekci proměnných. Prozkoumány jsou vztahy mezi jednotlivými příznaky.

