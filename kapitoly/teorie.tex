\chapter{}

\cite{bib:Jones}

\section{Logistika}

\subsection*{Definice Logistiky}

Logistika zahrnuje všechny operace, které se týkají doručení zboží nebo služeb od výrobce k zákazníkovi, s výjimkou samotné výroby zboží nebo provádění služby. Výrobou je naopak rozuměno vše, co mění podobu materiálu.
Během výroby se však logistika uplatňuje, například jako přesun materiálu nebo polotovarů mezi jednotlivými výrobními zařízeními. 
% Obdobně při poskytování služby je podstatné se zabývat 
Operace lze rozdělit do tří hlavních toků: materiálový, informační a finanční tok. Materiálový obsahuje všechny pohyby týkající se fyzického materiálu, tedy jeho získávání, přesuny a skladování, a to jak mezi zákazníky, dodavateli či výrobními areály a sklady, tak i vnitřní pohyby mezi produkčními linkami nebo skladovými pozicemi. Informační tok popisuje procesy vznikající během materiálového toku, dále se do něj řadí analýzy již proběhlých toků a plánování a předpovědi budoucích toků. Poslední kategorie, finanční tok mapuje náklady způsobené předešlými dvěma zmíněnými toky.\cite{bib:Baudin}

Pojem logistika je úzce propojen s pojmem Supply Chain Management (SCM)\footnote{Do češtiny lze Supply Chain Management přeložit jako řízení či správa dodavatelského řetězce. V českém prostředí se používá jak anglická tak česká podoba.}. Zatímco logistika se zabývá toky zboží, služeb či lidí, Supply Chain Management zahrnuje operace logistiky, navíc ale sleduje vztahy mezi procesory, které koordinuje a optimalizuje za účelem naplnění určitých cílů. Tímto cílem bývá často snížení nákladů v rámci částí procesu nebo zvýšení konkurenceschopnosti podniku \cite{bib:IIMudaipur}. Supply Chain Management se tedy prolíná s pojmem logistika a bývají často zaměňovány. Důvodem může být i to, že se jedná o nový pojem, který byl poprvé použitý v roce 1982.\cite{bib:Christopher} 

\section{Štíhlá logistika}

Štíhlá logistika nachází svůj původ na začátku 20. století, kdy henry Ford zavedl pohyblivou montážní linku při výrobě automobilu Ford modelu T. Tato linka měla za následek několikanásobné snížení výrobního času a odstartovala sériovou výrobu aut. Díky čemuž se snížila prodejní cena, a  automobily tak byly dostupné nejen nejbohatší vrstvě společnosti. 
Po druhé světové válce navázala automobilová společnost Toyota Motor Company na Fordovu efektivní montážní linku a vytvořila systém nazvaný Toyota Production System (TPS), který je přímým předchůdcem štíhlé logistiky.\cite{bib:seven}

Toyota Production System je založen na pěti základních principech. Nejdůležitějším krokem je odstranit plýtvání. Je třeba se soustředit na jednotlivé procesy a na vazby mezi nimi. Pomocí metody genchi genbutsu\footnote{Genchi v překladu znamená skutečná lokace a genbutsu skutečná věc.} se nasbírají data a informace o procesech přímo na místě, kde procesy probíhají, aby případné problémy a zdroje plýtvání mohly být přesně určeny. Po této analýze se aplikuje přístup řešení problémů zvaný kaizen\footnote{Kaizen je japonský překlad slova zlepšení.}, jehož cílem je kontinuální zlepšování procesů. Posledním z principů je  dodržování vzájemného respektu mezi všemi oddělení společnosti, jak vedoucími pracovníky, tak zaměstnanci u výrobních linek. Tento princip podobně jako kaizen je přímo napojený na filosofii firmy.\cite{bib:seven}

Taiichi Ohno, manažer ve společnosti Toyota, identifikoval sedm podob plýtvání, někdy nazývané \emph{seven deadly wastes}.
\begin{enumerate}
    \item Nadprodukce -- 
    \item Zpoždění/čekání --
    \item Transport --
    \item Pohyb 
    \item Skladování
    \item Defekty
    \item 
\end{enumerate}

1. Overproduction: Producing more than is needed for immediate use. 
2. Delay/Waiting: Any delay between the end of one process and the start of the next activity. 
3. Transportation/Conveyance: Unnecessary movement of products, materials or information. 
4. Motion: Unnecessary movement of people, such as walking, reaching and stretching. 
5. Inventory: Any raw material, work-in-process, or finished goods that exceed what is required 
to meet customer needs just in time and to maintain process stability. 
6. Over-processing: Using more energy or activity than is needed to produce a product - or 
adding more value than the agreed standard.  
7. Defects/Correction: Any production that results in rework or scrap

\subsection*{MUDA}
\subsection*{MULA}