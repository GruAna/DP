\chapter{Logistika}

Tato kapitola se nejprve věnuje základním pojmům z~ logistiky, poté jsou definovány pojmy týkající se plýtvání v~ tomto oboru. Poslední část kapitoly obsahuje definici pojmu shrink a jeho dělení na různé typy. Představeno je teoretické rozdělení podle literatury, které je následně ilustrováno na datech vybrané společnosti.

\section{Definice logistiky}

Logistika zahrnuje všechny operace, které se týkají doručení zboží nebo služeb od výrobce k~zákazníkovi, s~výjimkou samotné výroby zboží nebo provádění služby. Výrobou je naopak rozuměno vše, co mění podobu materiálu.
Během výroby se však logistika uplatňuje, například jako přesun materiálu nebo polotovarů mezi jednotlivými výrobními zařízeními. 
% Obdobně při poskytování služby je podstatné se zabývat 
Operace lze rozdělit do tří hlavních toků: materiálový, informační a finanční tok. Materiálový obsahuje všechny pohyby týkající se fyzického materiálu, tedy jeho získávání, přesuny nebo skladování, a to jak mezi zákazníky, dodavateli či výrobními areály a sklady, tak i vnitřní pohyby mezi produkčními linkami nebo skladovými pozicemi. Informační tok popisuje procesy vznikající během materiálového toku, dále se do něj řadí analýzy již proběhlých toků a plánování budoucích toků. Poslední kategorie, finanční tok, mapuje náklady způsobené předešlými dvěma zmíněnými toky.\cite{bib:Baudin}

Pojem logistika je úzce propojen s~pojmem Supply Chain Management (SCM)\footnote{Do češtiny lze Supply Chain Management přeložit jako řízení či správa dodavatelského řetězce. V~českém prostředí se používá jak anglická tak česká podoba.}. Zatímco logistika se zabývá toky zboží, služeb či lidí, Supply Chain Management zahrnuje operace logistiky, navíc ale sleduje vztahy mezi procesory, které koordinuje a optimalizuje za účelem naplnění určitých cílů. Tímto cílem bývá často snížení nákladů v~rámci částí procesu nebo zvýšení konkurenceschopnosti podniku \cite{bib:IIMudaipur}. Supply Chain Management se tedy prolíná s~pojmem logistika a často bývají bývají tyto pojmy zaměňovány. Důvodem může být i to, že SCM je nový pojem, který byl poprvé použitý v~roce 1982.\cite{bib:Christopher} 

\section{Štíhlá logistika}

Štíhlost neboli \emph{lean} je koncept neustálého vylepšování procesu vytváření produktu nebo služby pomocí odstranění jakéhokoli plýtvání. Plýtváním rozumíme jakoukoli činnost, která v~očích zákazníka nezvyšuje hodnotu produktu a tedy není ochotný za tuto činnost zaplatit ve formě vyšší prodejní ceny. Z~této definice plýtvání je patrné, že pohled zákazníka hraje důležitou roli při vytváření hodnoty produktu ve štíhlých systémech.\cite{bib:LW1,bib:LW2}

Svůj původ nachází štíhlá logistika na začátku 20. století, kdy Henry Ford zavedl pohyblivou montážní linku při výrobě automobilu Ford model T. Tato linka měla za následek několikanásobné snížení výrobního času a odstartovala sériovou výrobu aut. Díky čemuž se snížila prodejní cena, a tak automobily přestaly být dostupné jen pro nejbohatší vrstvě společnosti, ale i pro střední třídu. 
Po druhé světové válce navázala automobilová společnost Toyota Motor Company na Fordovu efektivní montážní linku a vytvořila systém nazvaný Toyota Production System (TPS), který je přímým předchůdcem štíhlé logistiky.\cite{bib:seven}

\subsection{Toyota Production System}

Toyota Production System je založen na pěti základních principech. Nejdůležitějším krokem je odstranit plýtvání. Je třeba se soustředit na jednotlivé procesy a na vazby mezi nimi. Pomocí metody \emph{genchi genbutsu}\footnote{Genchi v~překladu znamená skutečná lokace a genbutsu skutečná věc.} se sesbírají data a informace o procesech přímo na místě, kde procesy probíhají, aby případné problémy a zdroje plýtvání mohly být přesně určeny. Po této analýze se aplikuje přístup řešení problémů zvaný \emph{kaizen}\footnote{Kaizen je japonský překlad slova zlepšení.}, jehož cílem je kontinuální zlepšování procesů. Posledním z~principů je  dodržování vzájemného respektu mezi všemi odděleními společnosti. Tím jsou myšleni jak vedoucí pracovníci, tak zaměstnanci u výrobních linek.\cite{bib:seven}

V TPS je plýtvání rozděleno do tří kategorií -- Muda (plýtvání), Mura (nevyváženost) a Muri (přetěžování) \cite{bib:LW3}. V~následující části jsou podrobněji popsány jednotlivé typy.

\subsubsection*{Muda}

Japonské označení Muda v~překladu znamená plýtvání, neužitečnost či marnost. Muda zahrnuje všechny činnosti, které nepřispívají ke zvyšování hodnoty produktu. Mudu lze rozdělit na dva podtypy -- 1. typ zahrnuje aktivity, které jsou nezbytné pro koncového zákazníka, např. testování, zda je produkt nebo služba bezpečná. Druhý typ obsahuje ty procesy, které již zákazník nepotřebuje, či dokonce nechce, neboť mohou mít vliv na rychlost výroby produktu (výkonu služby) nebo přímo na jeho kvalitu.

Taiichi Ohno, manažer ve společnosti Toyota, identifikoval sedm typů plýtvání, někdy nazývané \emph{seven deadly wastes}. Klasifikace a popis včetně příkladů je uveden níže \cite{bib:seven}:
\begin{enumerate}
    \item \textbf{Nadprodukce} -- Pokud je vyrobeno více produktů, než je možné expedovat k~zákazníkovi, nebo více materiálu, než kolik je požadováno k~další výrobě či okamžité spotřebě.
    \item \textbf{Zpoždění/čekání} -- Jakákoli prodleva mezi dvěma na sebe navazujícími procesy, např. čekání jedné montážní linky na meziprodukty z~jiné linky vlivem rozdílných výrobních časů nebo vlivem nedostatečné výrobní kapacita jednoho ze strojů, dále sem patří také čekání zaměstnanců z~důvodu kontroly odvedené práce, pomalého načítání počítačového programu nebo čekání na konkrétní instrukce k~výkonu práce \cite{bib:LW1}.
    \item \textbf{Transport} -- Zbytečný přesun produktů, materiálů nebo informací. Tento transport navíc může vést k~poškození produktu. Příkladem tohoto typu plýtvání může být situace, kdy materiál, který je nejvíce potřebný pro výrobu produktů je umístěn v~největší vzdálenosti, nebo pokud přístup k~jedné položce ve skladu je blokovaný jinými položkami.
    \item \textbf{Pohyb} -- Zbytečný pohyb lidí, vzniklý špatným rozmístěním objektů v~prostoru, např. nepřiměřeně dlouhotrvající chůze, natahování se pro předměty, vyhýbání se lidem či předmětům. 
    \item \textbf{Skladování} -- Pokud je naskladněno více surovin, rozpracovaných výrobků a hotových produktů, než kolik je požadováno, např. předčasná dovážka položek do skladu, chyba v~dodávce, naskladnění položek do zásoby tzv. pro jistotu nebo z~důvodu množstevní slevy.
    \item \textbf{Nadbytečné zpracování} -- Při výrobě dochází k~použití více energie nebo prostředků než nutné, nebo je vytvořen koncový produkt, který má vyšší hodnotu, než jaký je dohodnutý a požadovaný standard. 
    \item \textbf{Defekty} -- Produkty či meziprodukty, které je nutné přepracovat nebo odstranit z~výroby z~důvodu vady. 
\end{enumerate}

Tyto podoby plýtvání aplikované v~TPS byly inspirací pro identifikaci sedmi typů plýtvání v~logistice \cite{bib:seven, bib:Jirsak}:
\begin{enumerate}
    \item \textbf{Nadprodukce} -- V~případě logistiky je nadprodukce chápána jako doručení produktů dříve nebo ve větším množství něž bylo požadováno.
    \item \textbf{Zpoždění/čekání} -- Jakákoli prodleva mezi dvěma na sebe navazujícími procesy, např. čekání na převoz meziproduktů mezi dvěma výrobními linkami, příjezd kamionu mimo časové okno, doba mezi příjezdem kamionu a jeho naložením nebo čas mezi přijetím objednávky a zahájením její realizace. 
    \item \textbf{Transport} -- Zbytečný přesun produktů, materiálů nebo informací, např. materiál, který je nejvíce potřebný pro výrobu produktů je umístěn v~největší vzdálenosti, nebo pokud přístup k~jedné položce ve skladu je blokovaný jinými položkami.
    \item \textbf{Pohyb} -- Zbytečný pohyb lidí, např. vzniklý špatnou organizací předmětů ve skladu, kdy položky, ke kterým se nejčastěji přistupuje, jsou v~méně přístupných pozicích skladu, nebo dokonce sklad není strukturovaný vůbec, nebo nutnost změnit trasu při převozu položek ve skladu kvůli nedostatečně širokým uličkám.
    \item \textbf{Skladování} -- Pokud je naskladněno více surovin, rozpracovaných výrobků a hotových produktů, než kolik je požadováno, např. předčasná dovážka položek do skladu, chyba v~dodávce, naskladnění položek do zásoby tzv. pro jistotu.
    \item \textbf{Prostor} -- Neoptimální využití dostupného místa, např. nedostatečná výška regálů ve skladech, nevyužitá kapacita regálů, neoptimální naložení kamionu, přetížení dostupných kapacit.
    \item \textbf{Defekty} -- Činnosti, které způsobí nutnost opakovat určitý proces, znehodnocení produktu nebo zvýší náklady, např. špatné zavezení produktu, špatné nebo chybějící označení produktu, chyby v~evidenci.
\end{enumerate}

V devadesátých letech, kdy se metody TPS začaly aplikovat ve společnostech, byl mezi sedm typů plýtvání Muda začleněn osmý typ - Dovednosti. V~tomto případě dochází k~neefektivitě kvůli nevyužití lidského potenciálu a talentů jednotlivých zaměstnanců. K~tomu může docházet například striktním rozdělením na manažery a zaměstnance, kde role zaměstnanců je poslouchat nařízení shora a vykonávat práci tak, jak byla navržena vedoucími pracovníky. Avšak právě zaměstnanci pracující přímo v~terénu lépe identifikují případné problémy a snadněji naleznou řešení díky svým zkušenostem.\cite{bib:LW1}


% Základním kamenem pro odstranění plýtvání v~TPS je koncept "Just-in-Time", při jehož aplikování se vyrábí pouze to, co je aktuálně potřeba v~přesně požadovaném množství.

\subsubsection*{Mura}

Mura lze přeložit jako nestejnoměrnost, nevyrovnanost a nepravidelnost. Jedná se o plýtvání vznikající špatnou provázaností jednotlivých procesů a to jak interních, tak externích. Následkem nevyváženosti je pak vznik plýtvání Muda. \cite{bib:LW3, bib:Jirsak}

Plýtvání v~podobě Mura se rozlišuje jak v~procesech informačního, tak hmotného toku. V~případě informačního toku je nejvýznamnějším zdrojem plýtvání situace, 
kdy je chybně predikována poptávka mezi jednotlivými články logistického řetězce. Ignorování vztahů mezi procesy může vést k~chybovosti i v~řádu desítek procent. Informace, jejichž opomíjeni způsobuje chybovost předpovídání poptávky, mohou být např. v~jaké fázi životního cyklu se výrobek nachází, plánování promoakcí nebo výrobní a logistická omezeni dodavatelů.
Další zdroj Mura v~informačním toku je nedostatečná znalost stavu zásob mezi dodavatelem a odběratelem. Následkem čehož dochází k~méně častým zavážkám avšak s~větším objemem, což vede k~vyšším pojistným zásobám ve skladech. Většinu zmíněných situací lze eliminovat aplikováním konceptu "Just-in-Time" do jednotlivých procesů.
Plýtvání také vzniká při administrativě, pokud nejsou vhodně standardizované dokumenty používané v~logistickém řetězci. Příkladem může být špatná evidence pohybů ve skladu či tvorba objednávek. Nesjednocenost v~administrativních procesech vede ke zpomalení navazujících činností nebo dokonce k~chybám, které způsobí nemožnost dokončení celého procesu. Pak je nutné vybrané procesy provést znovu a napravit chyby.\cite{bib:Jirsak}

Plýtvání v~hmotném toku je přímým důsledkem chyb vznikajících v~informačním toku. Lze identifikovat i takové zdroje plýtvání, které nesouvisejí přímo s~informačním tokem, a to například dodržování různých standardů přepravních prostředků na straně dodavatele a odběratele. To má pak za následek nadbytečné překládání materiálu do podoby, kterou druhá strana používá a se kterou je schopna následně efektivněji manipulovat.\cite{bib:Jirsak}

\subsubsection*{Muri}

Pojem Muri označuje přetěžování. Muri často vzniká při snaze zvýšit produktivitu a odstranit tak předešlé typy plýtvání, v~konečném důsledku může ale vést k~výrazně větší chybovosti i celkovému selhání. Přetíženi mohou být zaměstnanci, ale i stroje. V~obou případech vytížení na více než 100~\% se může projevit na snížení kvality výstupu.  Lidé mohou být méně pozorní a může docházet k~nehodám, které mohou v~menší či větší míře negativně ovlivnit i větší část logistického řetězce. Stroje mohou produkovat zmetkové výrobky, nebo může dojít k~jejich poškození až zničení.\cite{bib:Jirsak,bib:LW3}

\subsubsection*{Příklad plýtvání Muda, Mura a Mudi}

Všechny tři zmíněné typy plýtvání Muda, Mura a Muri jsou navzájem propojené. Tuto skutečnost je třeba brát v~potaz při řešení zefektivňování procesů a eliminaci plýtvání. Pro představu je uvedena následující situace. Společnost potřebuje zákazníkovi přivézt šest tun materiálu, uloženého ve stejných jednotunových kontejnerech. Možné způsoby řešení této úlohy jsou znázorněné na obr. \ref{obr:log:3M}.\cite{bib:LW3}

\begin{figure}[h!]
    \centering
    \includegraphics[width=0.9\textwidth]{obrazky/3M.jpg}
    \caption{Příklady plýtvání Muda, Mura a Muri při transportu šesti tun materiálu.\cite{bib:LW3}}
    \label{obr:log:3M}
\end{figure}

Nejjednodušší možností je naložit na jeden kamion veškerý požadovaný materiál. V~takovém případě společnost ušetří na počtu vozidel a eliminuje tak plýtvání přepravními prostředky, ušetří čas při nakládce a vykládce, protože není nutné obsluhovat více vozidel, zároveň . Na druhou stranu ale hrozí přetížení kamionu. Následkem přetížení se může zvýšit riziko nehody vozidla, firma může být pokutována nebo vozidlu nemusí být umožněn vjezd na určitá místa.

Opačným extrémem je použít tři kamiony, každý se dvěma tunami materiálu. Potom ale není efektivně využita dostupná kapacita a je patrné, že dochází k~mnoha druhům plýtvání typu Muda.

Třetí možností je využití dvou kamionů, kdy první je naložen čtyřmi a druhý dvěma tunami. Toto rozložení nepodléhá žádným pravidlům a patrně proces nakládky není dostatečně spjatý s~ostatními procesy nebo neprobíhá správný přenos informací o požadavcích mezi jednotlivými procesy. Nakládka a vykládka prvního velmi naloženého kamionu vyžaduje více času než druhého kamionu. Z~toho plyne, že buď není možné v~dostupném čase stihnout obsloužit první kamion a dochází k~přetížení, anebo v~případě druhého kamionu je zbude velké množství času a zaměstnanci zbytečně čekají. Z~této volby plyne, že plýtvání typu Mura může způsobit Mudu i Muru.\cite{bib:LW3}

Optimální řešení je naložit dva kamiony po třech tunách, což je jejich ideální kapacita. V~takovém případě společnost minimalizuje za daných podmínek všechny tři typy plýtvání. V~reálném světě jsou situace mnohonásobně komplexnější a ne vždy existuje jednoznačné optimální řešení, které je navíc snadno dosažitelné. Důležité ale je soustředit se na všechny tři typy současně, protože optimalizace pouze jednoho kritéria může způsobit jiný druh plýtvání nebo kolaps části systému. 

V roce 2011 bylo realizováno dotazníkové šetření Vysokou školou ekonomickou v~Pra-\\ze, které mapovalo, kolik procent logistických expertů se zabývá odstraněním zmiňo-\\vaných tří typů plýtvání. Plýtvání Muda se snaží odstranit z~logistických procesů 72~\% respondentů, Murou se zabývá 39~\% a plýtvání Muri řeší 30~\% dotazovaných.\cite{bib:Jirsak}

\subsection{Plýtvání v~logistických procesech}

Tato sekce se zabývá třinácti vybranými logistickými procesy z~hlediska plýtvání, jak jsou uvedeny v~knize dr. Petra Jirsáka \emph{Logistika pro ekonomy -- Vstupní logistika} \cite{bib:Jirsak}. Analýza vychází z~již zmíněného dotazníkového šetření z~roku 2011. Procesy jsou seřazeny na základě hodnocení respondentů v~pořadí od těch procesů, které jsou považovány nejvíce za plýtvání, k~těm, které měli podle respondentů nejnižší význam.
 
\subsubsection*{Reklamace}

Téměř 60~\% dotazovaných považuje proces reklamování zboží za plýtvání. K~reklamaci zboží zpravidla dochází pokud je zboží vadné z~důvodu chyby, která vznikla během procesu výroby nebo při přepravě. Kromě ztracených vynaložených nákladů na výrobu zboží a jeho následnou přepravu, vznikají navíc další náklady spojené s~administrativou reklamace. Reklamace by totiž měly přezkoumány, aby mohly být schváleny. Aby se snížily náklady na dodatečnou přepravu k~dodavateli, resp. výrobci zboží, v~některých případech se čeká na větší množství reklamovaných produktů.

Poněkud odlišným druhem plýtvání v~případě reklamace je čas zákazníků, kteří výrobek zakoupili, ale byli donuceni ho vrátit. To může vést k~nespokojenosti zákaz-\\níka a jeho přechodu ke konkurenční společnosti.

\subsubsection*{Manipulace}

Manipulace byla respondenty označena za nejméně hodnototovrný proces, a to pouhými 4~\% dotazovaných. Podle logistických exportů se jedná o nezbytný proces. Zlepšením procesu manipulace může dojít ke zlepšení kvality a zkrácení průběžné doby.

\subsubsection*{Skladování, příjem a výdej do a ze skladové plochy}

Proces skladování jedna třetina logistických expertů považuje za plýtvání, zatímco zbylé dvě třetiny jej hodnotí jako nezbytný. Tento proces lze zařadi jako plýtvání typu Muda, tedy že se jedná o zbytečné plýtvání. Ne ale každé skladování je zbytečné, např. něktěré výrobky potřebují určitý čas zrát nebo také není ve všech případech možné sladit konec výroby produktu s~poptávkou. Celý proces by se pak mohl stát nestabilní při náhlém zvýšení poptávky.        

Také je důležité brát v~potaz vzdálenost všech subjektů. Při velkých vzdálenostech je třeba mít zásoby, kdyby došlo k~výpadku nějakého druhu dopravy, aby koncoví zákazníci nepocítili ihned problém s~doručením zboží.

\subsubsection*{Nakládka, překládka, vykládka}

Opět se jedná podle většiny expertů za nezbytný proces. Tyto procesy mohou pomoci odstranit plýtvání Muda tím, že díky vyššímu počtu překládek budou dopravní prostředky správně vytížené a nebudou zavážet s~nevyužitou kapacitou. Prostory pro zlepšení v~těchto procesech se nabízí ve zkrácení doby čekání vozidla na nakládku, resp. vykládku. Dále lze plýtvání odstranit v~případě použití vhodné manipulační techniky. Špatná nebo příliš různorodá technika může vést i k~plýtvání Mura.

\subsubsection*{Cross-dockové operace}

Cross-docková centra umožňují plynule přesouvat zboží z~centrálního skladu dále, kam je třeba. Zboží se tak zbytečně neskladuje, zároveň dochází k~významné úspoře přepravních nákladů, jelikož jsou co nejlépe využity kapacity dopravních prostředků. Proces také umožňuje sloučit objednané zboží od více dodavatelů do jedné objednávky, tím se snižuje čekací doba na dodání zboží. Centra se zřizují na místech s~dobrou dostupností pro sklady i odběratele.

\subsubsection*{Administrativní úkony}

Administrativa je v~logistických procesech nezbytná. Zavedením vhodného informa-\\čního systému a odstranění papírové dokumentace, lze plýtvání v~administrativě snížit a zrychlit celý proces. 

\subsubsection*{Kontrola kvality a kvantity}

Pokud by existovala 100\% spolehlivost kvality výrobku a jeho bezpečné přepravy a také správná kompletace objednávek, mohl by tento proces být zrušen. Tuto spolehlivost lze zvýšit odstraněním problémů v~předešlých procesech. Samotný proces lze urychlit a zkvalitnit například použitím moderních technologií -- automatické rozpoznávání obrazu, laserová kontrola aj.

\subsubsection*{Konsolidace}

Konsolidací se rozumí sloučení více dodávek do jedné. Pokud by k~tomuto procesu nedocházelo, a byly prováděny pouze přímé, v~mnoha případech špatně vytížené závozy, náklady na dopravu by byly mnohonásobně vyšší a dodací čas také. Jedná se tedy o vhodný způsob pro redukci plýtvání Muda.


\subsubsection*{Doprava}

Pouze 12~\% respondentů označilo dopravu za plýtvání, jelikož zákazník dopravou nezíská žádnou užitnou vlastnost produktu. Vhodnou volbou dopravy a přepravních operací, např. cross-dock, lze výrazně snížit přepravní náklady. Je důležité ale nezvyšovat dopravní náklady špatnou volbou přepravního prostředku a jeho špatnou vytížeností.

\subsubsection*{Řízení pojistných zásob}

Držení pojistných zásob je typická ukázka plýtvání. Každý podnik by měl dávat přednost situaci, kdy spouští své procesy pouze ve chvíli, kdy je známa skutečná poptávka. Své procesy by měl mít podnik nastaven tak, že veškeré přípravy materiálů, výrobu a závoz stihne pro předpokládanou dodávku včas s~minimem chyb.

Skutečnost ale ukazuje, že poptávka se nechová stále stejně a mohou být případy, kdy její predikce selže. V~jiných případech dojde z~externích důvodů k~výpadku výroby, a proto je třeba mít pojistnou zásobu na skladě. Nicméně, tato pojistná zásoba by měla být správně vypočtena, aby nedocházelo k~postupnému navyšování zásob. Je důležité zvážit, že některé produkty s~časem podléhají zkáze a držení velké zásoby takových produktů může vést k~jejich následnému vyhození bez užitku.

\subsubsection*{Balení}

Obaly mají především manipulační, ochrannou a informační funkci. Proces balení je nezbytný v~logistice. Aby zboží dorazilo od dodavateli k~odběrateli v~nezměněné kvalitě, je nutné jo zabezpečit proti poškození, to vyžaduje jisté množství obalového materiálu. Pokud je zboží několikrát překládáno, je třeba, aby obal, ve kterém zboží je, překládku usnadňoval a nedocházelo tak ke zbytečným časovým ztrátám. V~neposlední řadě je zboží třeba označit, aby nemohlo být zaměněno s~jiným.

\subsubsection*{Plánování}

Správné plánování umožňuje snížení nákladů a redukci prostojů. Zároveň ale může znamenat plýtvání v~podobě velkého množství dat k~analyzování, tvorbě složitých podkladů. Největší úskalí v~plánování ale je, pokud nebyl proces navržen dostatečně pružný a není možný reagovat na náhlé výkyvy poptávky. To může způsobit pak nadzásobu, nebo naopak nedostatek.

V odpovědích respondentů se ale pouze 2~\% z~nich přiklonila, že se jedná o plýtvání. 
% 78~\% z~nich souhlasí s~konceptem just-in-time (JIT)\footnote{JIT -- } a 76~\% s~využitím metody kanban.

\subsubsection*{Sdílení informací}

Proces sdílení informací s~dodavateli a odběrateli jako jediný nebyl považován, podle respondentů, za plýtvání. Spolupráce mezi oběma subjekty totiž může vést k~redukci Mudy, protože sdílení informací umožňuje lépe reagovat na aktuální poptávku. Odběratelé tak mohou získat kvalitnější produkty a produkty, které lépe odpovídají jejich potřebám.

Do jisté míry také přenos informací mezi dodavateli a odběrateli může vést ke snížení nákladů. Například díky lepší synchronizaci výroby a expedice k~odběrateli s~jeho závozy. Což vede také ke snížení dodacích lhůt a tedy zvýšení dostupnosti zboží, což vede ke spokojenosti zákazníků.

\chapter{Shrink}

Cílem této práce je analyzovat shrinky produktů, které byly zaznamenány v datech dané společnosti, a zjistit příčiny jejich vzniku. V následující části je vysvětlen pojem shrink a popsány kategorie, které vybraná společnost rozeznává ve svých datech.

\section{Definice}

Slovem shrink se označuje ztráta zisku z neuskutečněného prodeje hotového produktu. Tento produkt je vyroben, či naskladněn, ale z nějakého důvodu nemohl být prodán zákazníkovi. Tímto důvodem může být například poničení produktu, jeho ztráta nebo prošlá doba spotřeby. Za shrink produktu lze označovat i stav, kdy cena produktu je neplánovaně snížena v důsledku zmíněných důvodů. Shrinkem je potom rozdíl plánované prodejní ceny a ceny, za kterou byl produkt skutečně prodán.\cite{bib:DefShrink}

% Cílem každé společnosti by mělo být tyto shrinky minimalizovat, protože kvůli shrinkům firma přichází o zisk. 


\section{Typy shrinků}

Vybraná společnost rozlišuje ve svých datech tři kategorie shrinku -- inventory, damages, price downs. Dále se budu věnovat popisu jednotlivých typů v rámci kategorií. Každý typ má přiřazeno jednoznačné identifikační číslo, podle kterého je zaznamenáván v databázi. Z důvodů anonymizace dat v práci nejsou uvedené přesné hodnoty těchto ID, namísto toho jsou uvedeny názvy, které  definují shrinky.

\subsection*{Damages}

Do kategorie damages (česky škody) jsou řazeny zbylé důvody k odstranění produktu z prodeje z důvodu degradace produktu. V následující tabulce \ref{tab:sh:dam} jsou vypsané všechny typy, které moho být evidovány.

\begin{table}[hbtp!]
    \caption{Přehled jednotlivých typů shrinků z kategorie damages.}
    \label{tab:sh:dam}
    \begin{tabular}{ p{4cm} p{10.5cm}}
         Název             & Popis \\
    \hline
                Poškození               & Odpis zboží, které bylo poškozené. Např. nedopečené, spálené, špatně vyrobené nebo poškozené zaměstnancem nebo zákazníkem (kdy nelze uplatnit reklamaci na zákazníka.)       \\
                Prošlé a zkažené zboží  & Odpis zboží, kterému prošla doba spotřeby (v případě výrobků, kde je datum uvedené), zkažené či shnilé zboží (ovoce, zelenina) nebo ztvrdlé pečivo.       \\
                Potravinová banka       & Odpis potravinářského zboží, které bylo darováno potravinovým bankám. Jedná se o produkty, které nebylo možné zařadit znovu do oběhu.    \\
                Zvířecí útulky          & Odpis potravinářského zboží, které bylo darováno do útulků zvířat. Jedná se o produkty, které nebylo možné zařadit znovu do oběhu.          \\
                Uznané zákaznické \par reklamace \strut  & Odpis zboží, které zákazník reklamoval a reklamace byla uznána, ale zároveň nelze toto zboží reklamovat u dodavatele.      \\
                Neupl. reklamace \par distribučního centra \strut   &  Odpis zboží, které fyzicky nedorazilo z distribučního centra a nebylo možné ho reklamovat z důvodu nesplnění limitu pro vytvoření reklamace na distribučním centru. Také obsahuje odpisy neprodaných EXIT položek po ukončení výprodeje.     \\
                Kompostéry              & Odpis zboží, které je prošlé nebo poškozené a které prodejna zlikviduje v kompostéru.       \\
                Poškozeno vnějšími \par vlivy \strut %(internal use (err)) 
                                        & Odpis zboží, které bylo poškozeno nebo zničeno vlivem třetí strany (výbuch, vytopení, poškození majetku vloupáním) nebo přírodními živly. Zboží se tedy na prodejně nenachází a nemůže proto být zlikvidováno.      \\
                Zničeno  rozmražením    &       \\
    \end{tabular}
\end{table}

\subsection*{Inventory}

Pojem inventory, který lze do češtiny přeložit jako inventář, sdružuje všechny shrinky týkající se změn v inventáři, tj. stavu zásob či inventury. V tabulce \ref*{tab:sh:inv} se nachází přehled všech evidovaných typů.

\begin{table}[hbtp!]
    \caption{Přehled jednotlivých typů shrinků z kategorie inventory.}
    \label{tab:sh:inv}
    \begin{tabular}{ p{4cm} p{10.5cm}}
     Název             & Popis \\
    \hline
              Inventura$+$             & Kladné připsání zboží během inventury.      \\
              Inventura$-$             & Záporné odepsání zboží během inventury.      \\
              Inventura              & Velká inventura skladu.     \\
              Oprava inventury       & Dodatečné opravy, které bylo třeba provést po dokončení velké inventury.      \\
              Částečná inventura     & Odpis, nebo naskladnění zboží při inventuře položek.      \\
              Neuznané reklamace \par distribučního centra  \strut &  Odpis zboží, které bylo fyzicky dodané distribučním centrem na prodejnu, ale prodejna jej vrátila, ale distribuční centrum vratku neuznalo.     \\
              Inventura              & Starší verze ID používaného pro inventuru.\\
              Partes merchandise     & Odpis prokazatelně ukradeného zboží nebo i ztraceného zboží.      \\
    \end{tabular}
\end{table}


\subsection*{Price down}



