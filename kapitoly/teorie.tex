\chapter{}

\cite{bib:Jones}

\section{Logistika}

\subsection*{Definice Logistiky}

Logistika zahrnuje všechny operace, které se týkají doručení zboží nebo služeb od výrobce k zákazníkovi, s výjimkou samotné výroby zboží nebo provádění služby. Výrobou je naopak rozuměno vše, co mění podobu materiálu.
Během výroby se však logistika uplatňuje, například jako přesun materiálu nebo polotovarů mezi jednotlivými výrobními zařízeními. 
% Obdobně při poskytování služby je podstatné se zabývat 
Operace lze rozdělit do tří hlavních toků: materiálový, informační a finanční tok. Materiálový obsahuje všechny pohyby týkající se fyzického materiálu, tedy jeho získávání, přesuny a skladování, a to jak mezi zákazníky, dodavateli či výrobními areály a sklady, tak i vnitřní pohyby mezi produkčními linkami nebo skladovými pozicemi. Informační tok popisuje procesy vznikající během materiálového toku, dále se do něj řadí analýzy již proběhlých toků a plánování a předpovědi budoucích toků. Poslední kategorie, finanční tok mapuje náklady způsobené předešlými dvěma zmíněnými toky.\cite{bib:Baudin}



\section{Štíhlá logistika}

Rozdělení 

\subsection*{MUDA}
\subsection*{MULA}