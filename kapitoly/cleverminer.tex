\chapter{Analýza pomocí nástroje cleverminer}
\label{ch:cleverminer}

Pomocí nástroje Cleverminer, který je dostupný jako knihovna pro jazyk Python, jsem provedla analýzu shrinků produktu. Pracovala jsem pouze se vzorem dat jednoho měsíce a s kategorií produktů \emph{Velmi čerstvé}. Princip metod, které se používají v knihovně je popsán v sekci \ref*{sec:Teorie:Guha}. Dataset jsem rozšířila o další sledované sloupce, které dávají do srovnání hodnotu shrinku a objem tržeb. Vytvořila jsem takto sloupce: podíl shrinku na celkových tržbách prodejny, podíl shrinku na tržbách shrinkovaného produktu na prodejně, podíl shrinku a tržeb v kategorii úrovně 1.

Metoda pracuje pouze s diskrétními hodnotami, proto bylo nutné kategorizovat sloupce s hodnotou shrinku, s množstvím shrinkovaných produktů a s jednotlivými podíly. Na obrázcích \ref*{} až \ref{} jsou zobrazené četnosti záznamů v kategoriích.

TODO ukázky kategorizace TBD % TODO ukázky kategorizace TBD

\subsection{Výsledky}